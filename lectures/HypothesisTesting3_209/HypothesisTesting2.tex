\documentclass{beamer}
%
% Choose how your presentation looks.
%
% For more themes, color themes and font themes, see:
% http://deic.uab.es/~iblanes/beamer_gallery/index_by_theme.html
%
\mode<presentation>
{
  \usetheme{Boadilla}      % or try Darmstadt, Madrid, Warsaw, ...
  \usecolortheme{beaver} % or try albatross, beaver, crane, ...
  \usefonttheme{default}  % or try serif, structurebold, ...
  \setbeamertemplate{navigation symbols}{}
  \setbeamertemplate{caption}[numbered]
  
} 

\usepackage{xcolor,colortbl}
\usepackage[english]{babel}
\usepackage[utf8x]{inputenc}
\usepackage{courier}
\usepackage{dsfont}
\usepackage{verbatim} 
\usepackage{enumerate}
\usepackage{tikz}
\usetikzlibrary{shapes.geometric, arrows}
\usepackage{multirow}
\usepackage{venndiagram}
\usepackage{epigraph} 
%\usepackage{xcolor}
\usepackage{makecell}

%\usepackage{enumitem}

\usepackage{hyperref}
\hypersetup{
    colorlinks=true,
    linkcolor=blue,
    filecolor=magenta,      
    urlcolor=cyan,
}

% R stuff!
\usepackage{listings}
\definecolor{codegreen}{rgb}{0,0.6,0}
\definecolor{codegray}{rgb}{0.5,0.5,0.5}
\definecolor{codepurple}{rgb}{0.58,0,0.82}
\definecolor{backcolour}{rgb}{0.95,0.95,0.92}

\lstdefinestyle{mystyle}{
    backgroundcolor=\color{backcolour},    
    commentstyle=\color{codegreen},
    keywordstyle=\color{black},
    numberstyle=\tiny\color{codegray},
    stringstyle=\color{codepurple},
    basicstyle=\ttfamily\footnotesize,
    breakatwhitespace=false,         
    breaklines=true,                 
    captionpos=b,                    
    keepspaces=true,                 
    numbers=left,                    
    numbersep=5pt,                  
    showspaces=false,                
    showstringspaces=false,
    showtabs=false,                  
    tabsize=2
}

\lstset{style=mystyle}


\setbeamertemplate{enumerate items}[default]
\setbeamertemplate{itemize item}[triangle]

%\setitemize{label=\usebeamerfont*{itemize item}%
%  \usebeamercolor[fg]{itemize item}
%  \usebeamertemplate{itemize item}}
\tikzstyle{startstop} = [rectangle, rounded corners, 
minimum width=3cm, 
minimum height=1cm,
text centered, 
draw=black, 
fill=red!30]

\tikzstyle{io} = [trapezium, 
trapezium stretches=true, % A later addition
trapezium left angle=70, 
trapezium right angle=110, 
minimum width=3cm, 
minimum height=1cm, text centered, 
draw=black, fill=blue!30]

\tikzstyle{process} = [rectangle, 
minimum width=3cm, 
minimum height=1cm, 
text centered, 
text width=3cm, 
draw=black, 
fill=orange!30]

\tikzstyle{decision} = [diamond, 
minimum width=3cm, 
minimum height=1cm, 
text centered, 
draw=black, 
fill=green!30]
\tikzstyle{arrow} = [thick,->,>=stealth]


\title[Introduction to Statistics]{Hypothesis Testing pt. 3}
\subtitle{More Types of Hypothesis Tests}
\author{Grinnell College}
\date{}

\graphicspath{{img/}}

\begin{document}

\begin{frame}
  \titlepage
\end{frame}

\begin{frame}{Hypothesis Test -- Single Proportion}
H$_0$: p = p$_0$ \vspace{4mm}

Under the null hypothesis we have:
\begin{equation*}
    Z := \frac{\hat{p} - p_0}{\sqrt{\frac{p_0(1-p_0)}{n}}} \sim \textbf{N(0,1)}
\end{equation*} \vspace{-4mm}
\begin{itemize}
    \item use qnorm() with value of Z
\end{itemize} \vspace{4mm}

\textbf{Conditions:}
\begin{itemize}
    \item Random Sample
    \item $n \times p_0 \geq 10$
    \item $n \times (1-p_0) \geq 10$
\end{itemize}
\end{frame}

\begin{frame}{Hypothesis Test -- Difference of Proportions}
H$_0$: $p_1 - p_2$ = 0 \vspace{4mm}

If $p_1 = p_2$, then both are estimating the same thing.
\begin{equation*}
    \text{Let} \hspace{3mm} \widehat{p}_{pool} = \frac{x_1 + x_2}{n_1 + n_2} = \frac{n_1\hat{p}_1 + n_2\hat{p}_2}{n_1 + n_2}
\end{equation*}

If we are simulating what the null hypothesis looks like, then
\begin{equation*}
    Z := \frac{(\hat{p}_1 - \hat{p}_2) - 0}{\sqrt{\frac{\widehat{p}_{pool}(1-\widehat{p}_{pool})}{n_1}+\frac{\widehat{p}_{pool}(1-\widehat{p}_{pool})}{n_2}}} = \frac{(\hat{p}_1 - \hat{p}_2)}{\sqrt{\widehat{p}_{pool}(1-\widehat{p}_{pool})(\frac{1}{n_1}+\frac{1}{n_2})}} \sim \textbf{N(0,1)}
\end{equation*} \vspace{-4mm}
\begin{itemize}
    \item use qnorm() with value of Z
\end{itemize}

\textbf{Conditions:}
\begin{itemize}
    \item Random Samples
    \item $n_1 \times \widehat{p}_1 \geq 10$ and $n_1 \times (1-\widehat{p}_1) \geq 10$
    \item $n_1 \times \widehat{p}_2 \geq 10$ and $n_1 \times (1-\widehat{p}_2) \geq 10$
\end{itemize}
\end{frame}

\begin{frame}{Hypothesis Test -- Single Mean}
H$_0$: $\mu = \mu_0$ \vspace{4mm}

The CLT says that $\bar{x} \sim$ N($\mu$, $\frac{\sigma}{\sqrt{n}}$) \vspace{4mm}

If we are simulating what the null hypothesis looks like $\rightarrow$ $\bar{x} \sim$ N($\mu_0$, $\frac{\sigma}{\sqrt{n}}$) \vspace{4mm}

Think back to standardizing. If we standardize a Normal distribution it becomes a Standard Normal distribution N(0,1). So...\vspace{2mm}

\begin{equation*}
    Z := \frac{\bar{x}-\mu_0}{\sigma / \sqrt{n}} \sim \textbf{N(0,1)}
\end{equation*}

If we define Z in this way, then we know it follows a Standard Normal distribution and we have a way to calculate p-values.
\begin{itemize}
    \item use qnorm() function with value of Z
\end{itemize}
\end{frame}

\begin{frame}{Hypothesis Test -- Single Mean}
H$_0$: $\mu = \mu_0$ \vspace{4mm}

Issue: We probably don't know $\sigma$ \vspace{2mm}

\begin{equation*}
    T := \frac{\bar{x}-\mu_0}{s / \sqrt{n}} \sim \textbf{t(df = n-1)}
\end{equation*}
\begin{itemize}
    \item use qt() function with value of T and df
\end{itemize}
\end{frame}

\begin{frame}{Hypothesis Test -- Single Mean}
H$_0$: $\mu = \mu_0$ \vspace{3mm}

\textbf{Conditions:}
\begin{itemize}
    \item Random Sample
    \item Normal population \textbf{OR} n $\geq$ 30
\end{itemize} \vspace{3mm}

\textbf{If $\sigma$ is known}:
\begin{equation*}
    Z := \frac{\bar{x}-\mu_0}{\sigma / \sqrt{n}} \sim \textbf{N(0,1)}
\end{equation*} \vspace{-4mm}
\begin{itemize}
    \item use qnorm() function with value of Z
\end{itemize} \vspace{2mm}

\textbf{If $\sigma$ is not known:}
\begin{equation*}
    T := \frac{\bar{x}-\mu_0}{s / \sqrt{n}} \sim \textbf{t(df = n-1)}
\end{equation*} \vspace{-4mm}
\begin{itemize}
    \item use qt() function with value of T and df
\end{itemize}
\end{frame}


\begin{frame}{Hypothesis Test -- Difference of Means}
H$_0$: $\mu_1 - \mu_2 = \mu_0$ = 0 \vspace{3mm}

\textbf{Conditions:}
\begin{itemize}
    \item Random Sample
    \item Normal population \textbf{OR} $n_1 \geq$ 30 and $n_2 \geq 30$
\end{itemize} \vspace{3mm}

\textbf{If $\sigma$ is known}:
\begin{equation*}
    Z := \frac{\bar{x}_1-\bar{x}_2}{\sqrt{\frac{\sigma^2}{n_1}+\frac{\sigma^2}{n_2}}} \sim \textbf{N(0,1)}
\end{equation*} \vspace{-4mm}
\begin{itemize}
    \item use qnorm() function with value of Z
\end{itemize} \vspace{2mm}

\textbf{If $\sigma$ is not known:}
\begin{equation*}
    T := \frac{\bar{x}_1-\bar{x}_2}{\sqrt{\frac{s_1^2}{n_1}+\frac{s_2^2}{n_2}}} \sim \textbf{t(df = min($n_1$, $n_2$) - 1)}
\end{equation*} \vspace{-4mm}
\begin{itemize}
    \item use qt() function with value of T and df
\end{itemize}
\end{frame}

\end{document}
