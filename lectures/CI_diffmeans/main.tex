\documentclass{beamer}
%
% Choose how your presentation looks.
%
% For more themes, color themes and font themes, see:
% http://deic.uab.es/~iblanes/beamer_gallery/index_by_theme.html
%
\mode<presentation>
{
  \usetheme{Boadilla}      % or try Darmstadt, Madrid, Warsaw, ...
  \usecolortheme{beaver} % or try albatross, beaver, crane, ...
  \usefonttheme{default}  % or try serif, structurebold, ...
  \setbeamertemplate{navigation symbols}{}
  \setbeamertemplate{caption}[numbered]
  
} 

\usepackage{xcolor,colortbl}
\usepackage[english]{babel}
\usepackage[utf8x]{inputenc}
\usepackage{courier}
\usepackage{dsfont}
\usepackage{verbatim} 
\usepackage{enumerate}
\usepackage{tikz}
\usepackage{multirow}
\usepackage{venndiagram}
\usepackage{epigraph} 
%\usepackage{xcolor}
\usepackage{makecell}

%\usepackage{enumitem}

\usepackage{hyperref}
\hypersetup{
    colorlinks=true,
    linkcolor=blue,
    filecolor=magenta,      
    urlcolor=cyan,
}

% R stuff!
\usepackage{listings}
\definecolor{codegreen}{rgb}{0,0.6,0}
\definecolor{codegray}{rgb}{0.5,0.5,0.5}
\definecolor{codepurple}{rgb}{0.58,0,0.82}
\definecolor{backcolour}{rgb}{0.95,0.95,0.92}

\lstdefinestyle{mystyle}{
    backgroundcolor=\color{backcolour},    
    commentstyle=\color{codegreen},
    keywordstyle=\color{black},
    numberstyle=\tiny\color{codegray},
    stringstyle=\color{codepurple},
    basicstyle=\ttfamily\footnotesize,
    breakatwhitespace=false,         
    breaklines=true,                 
    captionpos=b,                    
    keepspaces=true,                 
    numbers=left,                    
    numbersep=5pt,                  
    showspaces=false,                
    showstringspaces=false,
    showtabs=false,                  
    tabsize=2
}

\lstset{style=mystyle}


\setbeamertemplate{enumerate items}[default]
\setbeamertemplate{itemize item}[triangle]

%\setitemize{label=\usebeamerfont*{itemize item}%
%  \usebeamercolor[fg]{itemize item}
%  \usebeamertemplate{itemize item}}



\title[STAT-209]{Confidence Intervals}
\subtitle{Difference in Means}
\author{Grinnell College}
\date{October 18, 2024}

\graphicspath{{img/}}

\begin{document}

\begin{frame}
  \titlepage
\end{frame}

\begin{frame}
We saw when looking at histograms or boxplots we could compare means or medians to see if groups were different.
\begin{itemize}
    \item maybe try to estimate the \textit{difference} between 2 pop. means?
\end{itemize}
\vspace{15mm}

We just saw how to estimate a single pop. mean, let's take what we know to figure this out.
\end{frame}

\begin{frame}{Example -- Waggle Dance}
Honeybee scouts investigate new home or food source options; the scouts communicate the
information to the hive with a “waggle dance.”
\vspace{3mm}

Scientists took bees to an island with only two possible options for nesting: one of very high quality and one of low quality.
\vspace{3mm}

They recorded:
\begin{itemize}
    \item \underline{quality of the sites}
    \item \underline{number of times a bee performed the dance (circuits)}
    \item distance to the nesting sites
    \item duration of waggle dance
\end{itemize} \vspace{3mm}

\textbf{Research question}: How is the number of waggle circuits related to quality of a nesting site?
\begin{itemize}
    \item estimate the difference in pop. mean number of waggle circuits for each nesting site
\end{itemize}
\end{frame}

\begin{frame}{Notation}
2 groups $\rightarrow$ need to keep track of info separately for each of them \vspace{2mm}

\textbf{Group 1:}
\begin{itemize}
    \item $\mu_1$ = pop. mean for group 1
    \item $\overline{x}_1$ = sample mean for group 1
    \item $s_1$ = std. dev. for group 1
    \item $n_1$ = sample size for group 1
\end{itemize} \vspace{3mm}

\textbf{Group 2:}
\begin{itemize}
    \item $\mu_2$ = pop. mean for group 2
    \item $\overline{x}_2$ = sample mean for group 2
    \item $s_2$ = std. dev. for group 2
    \item $n_2$ = sample size for group 2
\end{itemize} \vspace{4mm}

\textbf{Note:} Sometimes we may use A/B for subscripts or use letters that include more context about the groups
\end{frame}

\begin{frame}{CI for Difference in Means}
Our \textbf{point estimate} for $\mu_1 - \mu_2$ is unsurprisingly $\overline{x}_1 - \overline{x}_2$ \vspace{6mm}

Our \textbf{SE} formula is more complicated: \vspace{-4mm}

\begin{align*}
\sqrt{\frac{\sigma_1^2}{n_1} + \frac{\sigma_2^2}{n_2}}
\end{align*} \vspace{6mm}

Our \textbf{df} value is a little different too
\begin{itemize}
    \item df = min($n_1$, $n_2$) - 1
\end{itemize}
\end{frame}


\begin{frame}{CI for Difference in Means}
Putting this all together... \vspace{2mm}

\textbf{95\% CI for difference in population means:}
\begin{align*}
\overline{x}_1 - \overline{x}_2 \pm t_{(.975, df)} \times \sqrt{\frac{\sigma_1^2}{n_1} + \frac{\sigma_2^2}{n_2}}
\end{align*} \vspace{6mm}

\textbf{100(1-$\alpha$)\% CI for difference in population means:}
\begin{align*}
\overline{x}_1 - \overline{x}_2 \pm t_{(1-\alpha/2, df)} \times \sqrt{\frac{\sigma_1^2}{n_1} + \frac{\sigma_2^2}{n_2}}
\end{align*} \vspace{4mm}

\textbf{Note:} df = min($n_1$, $n_2$) - 1
\end{frame}

\begin{frame}{Difference in Means -- Interpretation}
\textbf{CI Interpretation} is a little more involved when we are looking for a difference in means

\begin{itemize}
    \item include context
    \item mention in some way the order we are comparing means
    \item a positive value for the CI indicates $\mu_1$ is larger than $\mu_2$
    \begin{itemize}
        \item $\mu_1 - \mu_2 > 0 \rightarrow \mu_1 > \mu_2$
    \end{itemize}
    \item a negative value for the CI indicates $\mu_1$ is larger than $\mu_2$
    \begin{itemize}
        \item $\mu_1 - \mu_2 < 0 \rightarrow \mu_1 < \mu_2$
    \end{itemize}
\end{itemize} \vspace{6mm}

"We are 100(1-$\alpha$)\% confident that (the difference in population means) is between (lower value) and (upper value)."
\end{frame}

\begin{frame}{Difference in Means -- Interpretation}
"We are 100(1-$\alpha$)\% confident that (the difference in population means) is between (lower value) and (upper value)." \vspace{10mm}

Example: Suppose we have a 90\% CI of (-12.3, 24.8) \vspace{2mm}

"We are 90\% confident that the difference in population means is between -12.3 and 24.8." \vspace{2mm}

\textbf{OR} \vspace{2mm}

"We are 90\% confident that the $\mu_1$ is between 12.3 \textit{lower} and 24.8 \textit{higher} than $\mu_2$" \vspace{2mm}

\textbf{OR} \vspace{2mm}

"We are 90\% confident that the pop. mean for group 1 is between 12.3 \textit{lower} and 24.8 
\textit{higher} than the pop. mean for group 2."
\end{frame}

\begin{frame}{Difference in Means -- Conditions}
In order to make a 100(1-$\alpha$)\% CI for the difference in pop. means we need the following to all be true:
\begin{itemize}
    \item There was a random sample for both groups
    \item $n_1 \geq 30$
    \item $n_2 \geq 30$
    \item the \textit{groups} must be independent of each other
    \begin{itemize}
        \item ask: do the values from one group influence values for another?
        \item this is not the same thing as saying both groups behave differently
    \end{itemize}
\end{itemize}
\end{frame}

\end{document}

